\documentclass[12pt]{article}
\usepackage[spanish]{babel}
\usepackage[utf8]{inputenc}
\title{Tarea 1 - Problema 1 - Programación Avanzada}
\author{Nicolás Aylwin}
\date{}
\begin{document}
\maketitle
El programa se encarga de obtener la posición del centro de masas de un sistema discreto de partículas. También calcula la masa total. Para esto utiliza la conocida fórmula:
\begin{equation}
\vec{r}_{cm}=\frac{1}{M}\sum_{i}^{n}m_{i}\vec{r_i} 
\end{equation}
Donde $n$ es la cantidad de partículas, $\vec{r}_{cm}$ la posición del centro de masa, $M$ la masa total, $m_i$ las masas de cada una de las partículas y $\vec{r_i}$ sus respectivas posiciones.

El método utilizado es principalmente el uso de dos ciclos `for'; uno para almacenar la información que entregue el usuario a través de la terminal, y el otro para calcular lo mencionado usando la Ec.(1). La información se almacena en una matriz de `$n$ x $3$', donde cada vector de tres dimensiones almacena posición en $x$, en $y$ y la masa de una partícula en específico. Esta misma se utiliza en el segundo ciclo para poder calcular lo buscado.

Luego del cálculo se vuelcan los resultados en pantalla de la terminal y así termina la ejecución del programa.

Cabe destacar que el ingreso de los datos es a través del comando `cin', por lo que se admiten como separaciones de distintas variables; tabulaciones, espacios. Y con enter se ingresan al programa.
\end{document}