\documentclass[12pt]{article}
\usepackage[spanish]{babel}
\usepackage[utf8]{inputenc}
\usepackage{multicol}
\usepackage{amsmath,amsfonts,amssymb}
\usepackage[left=2cm,top=2cm,bottom=2cm,right=2cm]{geometry}
\title{Tarea 4 - Problema 1 - Programación avanzada}
\author{Nicolás Aylwin}
\date{}
\begin{document}
\maketitle
Buscamos generar los Polinomios de Legendre a través del algoritmo de Gram-Schmidt. Estos se caracterizan por ser ortogonales entre sí en el intervalo $[-1,1]$, y porque su norma al cuadrado (o equivalentemente, el producto interno con sigo mismo) es:
\begin{align}
\langle P_l(x),P_l(x)\rangle =||P_l(x)||^2=\frac{2}{2l+1}
\end{align}
Con $l$ como el grado del polinomio. Además el producto interno entre dos polinomios se define como:
\begin{equation}
\langle p(x),q(x) \rangle =\int_{-1}^{1}{p(x)q(x)dx}
\end{equation}
Por otro lado, el algoritmo de Gram-Schmidt se basa en la obtención de vectores ortogonales a otros vectores. Se representa como (siendo $u_i$ cada vector generado por el proceso y $v_i$ un vector cualquiera):
\begin{align*}
u_1&=v_1 \\
u_2&=v_2-proj_{u_1}(v_2) \\
u_3&=v_3-proj_{u_1}(v_3)-proj_{u_2}(v_3) \\
\end{align*}
Se continua con este patrón para obtener todos los vectores ortogonales entre sí que uno quiera. Además se tiene que:
\begin{equation}
proj_{u}(v)=\frac{\langle u,v\rangle}{\langle u,u\rangle}u
\end{equation}
Si tenemos los dos primeros Polinomios de Legendre, respectivamente $P_0(x)=1$ y $P_1(x)=x$, podemos usar el algoritmo de Gram-Schmidt comenzando de un polinomio genérico del siguiente grado, como $p(x)=x^2$, y aplicar el algoritmo con el producto interno definido como en Ec.(2). Supongamos obtenemos un polinomio $Q_2$ que definitivamente es ortogonal a $P_0$ y $P_1$, pero en general no cumplirá la Ec.(1). Necesitamos normalizar. Sea para esto $c$ una constante cuqluiera. Podemos imponer que:
\begin{align*}
c\langle Q_2,Q_2 \rangle =c\int_{-1}^{1}{Q_2 Q_2dx}=\int_{-1}^{1}{(\sqrt{c}Q_2)(\sqrt{c}Q_2)dx}=\frac{2}{2\cdot 2+1}
\end{align*}
Así de forma general se tiene que $\sqrt{c}Q_l(x)=P_l(x)$ con 
\begin{align*}
c=\frac{2}{2l+1}\frac{1}{\langle Q_l,Q_l \rangle}
\end{align*}
Siendo $Q_l$ el polinomio obtenido con el algoritmo de grado $l$ y $P_l$ el polinomio de Legendre de tal grado.

Dejando claro esto, vamos al programa:

Los archivos que permiten esto son `poli.h' y `problema1.cc'. El primero es una clase de templates de polinomios, bastante genérica, lo único con sentido mencionar es un constructor que recibe el grado del polinomio y un vector. Este se espera en orden de mayor a menor grado, y de no tener los elementos suficientes rellena con ceros. Esto es particularmente útil para definir polinomios como $x^{cualquier grado}$ rápidamente, y es el constructor usado al crear cualquier polinomio. El segundo programa es el que en sí realiza lo pedido.

Se comienza en este definiendo la función `prod\_interno' como el producto interno entre dos polinomios (entregados const y por referencia). Se define tal como se expresa en la Ec.(2). Luego la función `proy' como proyección de un polinomio sobre otro.

Despues se pasa directo al main. En este se comienza definiendo una variable double `c' para normalizar, un vector de double con solo un uno (`v') y los dos primeros polinomios de legendre ($P_0$ y $P_1$) usando `v'. Se crea también un vector de polinomios llamado `w' y se almacenan $P_0$ y $P_1$ en este. Luego se da un mensaje sobre los polinomios encontrados.

Ahora comienza el ciclo for importante. De 2 a 20 iterando con `i', comienza creando un polinomio de double de grado `i' genérico ($x^i$), llamado `p\_i'. También se crea un polinomio `acum'. Luego se entra a otro for que itera sobre el vector de polinomios `w'. Aquí se acumula en `acum' el resultado de la proyección de `p\_i' sobre todos los polinomios almacenados en `w' (para asegurarnos que sea ortogonal a todos). Luego se sale de este ciclo y se resta `acum' a `p\_i', asignando este valor a `p\_i' (ahora es el `$Q_l$' mencionado antes). Seguido se normaliza usando $c$ y considerando $i=l$. Por último se almacena este nuevo Polinomio de Legendre en `w', se imprime a pantalla y se repite el ciclo.

Luego de imprimir todos se ponen a mano (uf) todos los Polinomios de Legendre según mathematica del grado 2 al 20. Seguido se imprime a pantalla el error entre los de grados equivalentes (obtenidos y tabulados) según:
\begin{align*}
\int_{-1}^1{(P_l(x)-p_l(x))^2}
\end{align*}
Con $P_l$ el Polinomio de Legendre obtenido y $p_l$ el entregado por mathematica.

Se incluye en la carpeta también un makefile para facilitar la evaluación.
\end{document}