\documentclass[12pt]{article}
\usepackage[spanish]{babel}
\usepackage[utf8]{inputenc}
\usepackage{multicol}
\usepackage{amsmath,amsfonts,amssymb}
\usepackage[left=2cm,top=2cm,bottom=2cm,right=2cm]{geometry}
\title{Tarea 3 - Problema 1 - Programación avanzada}
\author{Nicolás Aylwin}
\date{}
\begin{document}
\maketitle
Buscamos los complejos $C=a+bi$ con $a, b \in [-2,2]$ tal que se cumpla que en la siguiente iteración:
\begin{align*}
z_0&=0\\
z_{n+1}&={z^{m}}_n+C
\end{align*}
El módulo de $z_n$ sea menor a 2 en cada iteración luego de 100 de estas.

Para lograr esto, se creó un pequeño programa; consta de dos funciones, siendo una de estas `main'. La primera llamada `itera', recibe un complejo (constante y por referencia, para tener cuidado de cambiarlo, y mejorar el rendimiento respectivamente) y un entero. Estos corresponden a $C$ y $m$. En esta función se realiza la recurrencia en un ciclo `while' con la condición sobre $z_n$. De romperse esta, se entrega un 1, dando a entender que con ese $C$ y esa $m$ la recurrencia diverge. Dentro del bucle también se suma un contador por cada iteración, que al llegar a 100 es evaluado por un `if' para devolver un 0, dando a entender que con ese $C$ y esa $m$ convergió (o al menos no rompió la condición en 100 iteraciones).  

En `main' primero se le pide al usuario el valor de $m$, este es almacenado en una variable tipo `int' y esta a su vez convertida a `string' para crear un archivo de texto de nombre `no\_ divergen\_ m.txt'. Además se crean un complejo `c' y un recipiente de `int' llamado `res'. Luego se entra a dos ciclos `for' anidados, para recorrer toda la grilla en que puede estar $C$. 

Se crea un $C$, se introduce en `itera' junto al $m$ ingresado por el usuario, se recibe en `res' y se estudia el valor de este. De ser cero se incluye la parte real y la imaginaria de $C$ en el archivo .txt, luego se avanza un $1/10000$ en la parte real y se repite, al recorrer todos los valores de la parte real se avanza lo mismo pero en la parte imaginaria y se repite el proceso. 

Dado esto, el programa demora varios minutos en terminar de calcular, por esto se imprime a pantalla cada $C$ a analizar, así darle a entender al usuario que si están pasando cosas y no se quedo pegado su computador simplemente.

Luego de generar los archivos para $m=\{1,2,3,4\}$, se graficaron estos en formato .png con Gnuplot. Solo se incluyen el ejecutable, el archivo de texto que genera el ejecutable y los gráficos, pues cada .txt (excepto el de $m=1$) pesan alrededor de 200mb.
\end{document}