\documentclass[12pt]{article}
\usepackage[spanish]{babel}
\usepackage[utf8]{inputenc}
\usepackage{multicol}
\usepackage{amsmath,amsfonts,amssymb}
\usepackage[left=2cm,top=2cm,bottom=2cm,right=2cm]{geometry}
\title{Tarea 2 - Problema 1 - Programación avanzada}
\author{Nicolás Aylwin}
\date{}
\begin{document}
\maketitle
El primer desafío resulta en encontrar el valor de las funciones de Bessel de primer orden ($J_n(x)$). Sabiendo que estas cumplen la recurrencia:
\begin{equation*}
J_{n+1}(x)=\frac{2n}{x}J_n(x)-J_{n-1}(x)
\end{equation*}
Pero que esta recurrencia diverge de los valores reales para $n$ creciente, se utiliza el algoritmo de Miller, en la que se utiliza la recurrencia decreciente. Matemáticamente:
\begin{equation*}
J_n(x)=\frac{2(n+1)}{x}J_{n+1}(x)-J_{n+2}(x)
\end{equation*}
Se programó un ciclo `for' que realizara esta recurrencia para un $n$ cualquiera, comenzando desde $N\approx n+\sqrt{10n}+1$ (solo busqué más precisión de la sugerida). De obtener un $N$ menor a 20, se ajustó este a 20.

Durante el ciclo, se acumula la suma de $2J_i$ con $i$ par, y $J_0$. También se almacena en una variable el $J_n$ con el $n$ pedido en la llamada de la función.

Luego de terminar el ciclo, se divide la suma mencionada al valor de $J_n$ almacenado, y se devuelve este último como valor de $J_n(x)$ para $x$ y $n$ dado. Esta normalización viene de la condición que cumplen las funciones de Bessel:
\begin{equation*}
1=J_0(x)+\sum_{i=1}^{\infty}2J_{2i}(x)
\end{equation*}
No son necesarias `tantas' funciones de Bessel para lograr un valor muy cercano pues a mayor $n$ en $J_n$, la función evaluada se acerca a cero rápidamente. Por esto se logra una buena aproximación con pocos valores.

Luego se busca graficar $f(x)=J_0(x)+2\sum_{i=1}(-1)^{i}J_{2i}(x)$ y $g(x)=2\sum_{i=0}(-1)^iJ_{2i+1}(x)$. Para esto se crea una función que entregue un vector conteniendo todas las $J_n(x)$ con $0\leq n\leq N$ para un $x$ y $N$ dado. Se crea también una función que separa de este último vector, los $J_n$ pares e impares, los almacena en vectores separados para así entregarlos a $f$ y $g$ respectivamente.

Estas últimas se programaron para recibir los vectores mencionados y retornar un valor, correspondiente a la función evaluada en algun $x$.

Usando todo esto, se establece un ciclo `for' en el `main' que evalua en 100 puntos entre $0$ y $2\pi$ el valor de $f$ y $g$, guardando en cada iteración el valor de $x$ y su evaluación en  `grafico\_f.dat' y `grafico\_g.dat' respectivamente. El programa se ejecuta, hace todo, y luego de crear estos dos archivos, lanza un mensaje a la terminal indicando el éxito del programa.

Luego con estos archivos se crean gráficos a través de Gnuplot, almacenados en esta misma carpeta. En particular, $f(x)$ se plotea junto a $\cos(x)$, y $g(x)$ con $\sin(x)$, mostrándose así que al menos en este rango son la misma función.
\end{document}
%Para cada problema, usted debe generar un único texto en L A TEX, en el que explique de
%manera clara y breve el método utilizado para desarrollar su programa, modo de uso
%y los resultados del mismo (output de programas, gráficos, ecuaciones, tablas, etc.) si es
%necesario.
